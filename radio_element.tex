\documentclass[12pt]{article}
%\usepackage[utf8]{inputenc}
\usepackage[margin=1in]{geometry}
\setlength{\parindent}{0pt}
\usepackage{filecontents}


\title{Tagesberichtdienst und Roter Faden}


\author{G1 Team 2019}



\date{July 2019}
% Tikz settings optimized for causal graphs.
% Just copy-paste this part

\begin{document}

\maketitle

Dieses Jahr haben wir uns beim Vob darauf geeinigt einen Tagesberichtdienst einzufuehren, der in irgendeiner Form die Geschehnisse des Tages kreativ verarbeiten sollte. 
Ein paar Menschen haben sich jetzt kurz vor dem Lager noch einmal ueberlegt wie das genau aussehen koennte und wie wir es darueberhinaus in den roten Faden integrieren koennten. 
Daraus ist folgende Idee enstanden: 


In der Gruppe der Mesnchen die in die Vereinigten Staaten auf der Suche nach einem besseren Leben einwandern befindet sich auch eine Gruppe von jungen Journalist*innen. 

Diese beschlie\ss en am zweiten Tag eine Radiosendung ueber die neue Gemeinschaft auf die Beine zu stellen. 
In dieser Radiosendung soll ueber wichtige Themen in der community berichtet bzw. kritisch reflektiert werden. 

Die Teilnehmer*innen nehmen dann abwechseln je nach Einteilung des Dienstes fuer einen Tag die Rolle dieser Journalist*innen ein und produzieren gemeinsam eine kleine Radiosendung. 
Diese Radiosendung koennte zum Beispiel ueber das Theater am selben Tag berichten. Optimalerweise waere es ein Format in dem kurz ueber die Inhalte des Theaters berichtet wird und diese dann auf ḱreative Art und Weise diskutiert werden. 

Im Laufe des Lagers wird es aber immer schwerer fuer die Journalist*innen ihre kritische Sendung zu produzieren da populitsische Kraefte versuchen die Krise in der zweiten Woche zu instrumentalisieren. 
Im Zuge dessen sollen auch kritischen Stimmen mundtot gemacht werden. 


Praktisch koennte der Radio Dienst dann in der zweiten Woche an das Lagerparlament uebergeben werden und von da an selber von diesem im Untergrund organisiert werden.
Dabei koennten manche Betreuer*innen ihnen das Leben zunehmend schwer machen. 
Dadurch wuerden wir auch nicht zu viele Teilnehmer*innen dazu zwingen einen Radiobeitrag zu drehen. 

Das Gelaendespiel koennte dann den Versuch der Journalist*innen darstellen mit einer großen Kampagne im Untergrund die populistischen Kraefte zu besiegen. 
Da koennen wir uns allerdings auch waehrend dem Lager je nachdem wie gut das Ganze läuft noch Gedanken machen. 

Wichtig ist noch dass wir damit das gut funktinoiert noch einen expliziteren Boesewicht bzw. Antagonisten brauchen.  
Optimalerweise wuerden diese Boesewichte schon waehrend der ersten Woche auftauchen und versuchen die Gemeinschaft gegeneinander aufzuhetzen um selber mehr Macht zu erlangen. In der zweiten Woche im Zuge der großen Krise wuerden sie dann immer erfolgreicher werden und die Gemeinschaft zunehmend ihren eigenen Interessen enstprechend umgestalten.    


\end{document}